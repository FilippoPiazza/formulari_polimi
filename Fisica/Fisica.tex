\documentclass[8pt]{scrreprt}
\usepackage[cm]{fullpage}
\usepackage{amsmath}
\usepackage{amssymb}
\usepackage{mathtools}
\usepackage{cases}
\usepackage{multicol}
\usepackage{geometry}

\pagestyle{empty}

\newgeometry{left=1cm,bottom=1cm, right=1cm, top=1cm} %margini ridotti
\begin{document}
\begin{multicols*}{3}

\section*{Cinematica}
Corpo in caduta da fermo:\\
$v = \sqrt{2gh}$\\
$t = \sqrt{\frac{2g}{h}}$

\vspace{-3mm}
\subsection*{Proiettile}
$y = x \tan \alpha - \frac{gx^2}{2v_0^2 \cos^2 \alpha}$\\
$h_{max} = \frac{v_0^2 \sin^2 \alpha}{2g}$\\
$x_{max} = \frac{v_0^2 \sin 2\alpha}{g}$\\
$\theta = \tan^{-1} \left(-\frac{2hg}{v_0}\right)$\\
$v_0 = \sqrt{\frac{dg}{\sin 2\alpha}}$

\vspace{-3mm}
\subsection*{Moto circolare}
Velocità angolare: $\omega = \frac{d\alpha}{dt}$\\
Accel. angolare: $\alpha = \frac{d\omega}{dt} = \frac{d^2\alpha}{dt^2}$

\vspace{-3mm}
\subsection*{Moto circolare uniforme}
$\omega = \frac{2\pi}{T}$\\
$v_{tang} = \omega r$\\
$a_{centr} = \frac{v^2}{r} = \omega^2 r$\\
$a_{tang} = \frac{dv}{dt}= \alpha R$

\vspace{-3mm}
\subsection*{Moto circolare unif. accel.}
$\omega = \alpha t$\\
$v = \omega r = \alpha r t$

\vspace{-3mm}
\subsection*{Moto curvilineo}
$\overrightarrow{a} = a_T \hat{\theta} +a_R\hat{r} = \frac{d|\overrightarrow{v}|}{dt}\hat{\theta} - \frac{v^2}{r}\hat{r}$

\vspace{-3mm}
\subsection*{Sistemi a più corpi}
$\overrightarrow{r}_{CM} = \sum_i \frac{m_i \overrightarrow{r}_i}{M} = \frac{1}{M} \int \overrightarrow{r} dm$\\
$\overrightarrow{v}_{CM} = \frac{d\overrightarrow{r}_{CM}}{dt} = \frac{1}{M} \sum_i m_i \overrightarrow{v}_i = \frac{1}{M} \int \overrightarrow{v} dm$\\
$\overrightarrow{a}_{CM} = \frac{d\overrightarrow{v}_{CM}}{dt} = \frac{1}{M} \sum_i m_i \overrightarrow{a}_i = \frac{1}{M} \int \overrightarrow{a} dm$\\
Momento di inerzia:\\
$I_{asse} = \sum_i m_i r_i^2 = \int r^2 dm$\\
Teorema assi paralleli:\\
$I_{asse} = I_{CM} + Md^2$

\vspace{-3mm}
\subsection*{Forze, lavoro, energia}
Gravità: $\overrightarrow{F} = -\frac{GMm}{r^2}\hat{r}$\\
Elastica: $\overrightarrow{F} = -k\overrightarrow{x}$\\
\underline{Attrito:}\\
Statico: $F_{max} = \mu_s N$\\
Dinamico: $F = \mu_d N$\\
\subsection*{Lavoro:}
$L=\int_{x_i}^{x_f} \overrightarrow{F} \cdot d\overrightarrow{l}= \int_{\theta_i}^{\theta_f} \tau d \omega$\\
Forza costante: $L = \overrightarrow{F} \cdot \overrightarrow{l}= Potenza \cdot\Delta t$\\
Forza Elastica:\\
$L=-\frac{1}{2} k x^2$\\
Forza peso: $L = -mgh$\\
Forza Gravità: $L = -\frac{GMm}{\Delta r}$

\vspace{-3mm}
\subsection*{Energia}
Cinetica: $K = \frac{1}{2}mv^2$\\
Rotazionale: $K = \left\{ \begin{array}{ll} \frac{1}{2}I\omega^2 & \text{asse} \\ \frac{1}{2}mv^2 + \frac{1}{2}I\omega^2 & \text{CM} \end{array} \right.$\\
Forze vive: $K_f - K_i = L_{TOT}$\\
Potenziale: $U = -L = -\int \overrightarrow{F} \cdot d\overrightarrow{l}$\\
Meccanica: $E = K + U = \frac{1}{2}mv^2 + U$\\
Conservazione: $E_f - E_i = L_{NC}$\\
Potenziale Elastica: $U = \frac{1}{2}kx^2$

\vspace{-3mm}
\section*{Impulso e Momento angolare}
Quantit\`a di moto: $\overrightarrow{p} = m\overrightarrow{v}$\\
Impulso: $\overrightarrow{I} = \int \overrightarrow{F} dt = \Delta \overrightarrow{p}$\\
Momento angolare: $\overrightarrow{L} = \overrightarrow{r} \times \overrightarrow{p}$\\
Intorno ad asse fisso: $ |\overrightarrow{L}| = I \omega$

\vspace{-3mm}
\subsection*{Equazioni cardinali}
$\overrightarrow{p}_T = \sum \overrightarrow{p}_i = m_{tot} \overrightarrow{v}_{CM}$\\
$\overrightarrow{L}_T = \sum \overrightarrow{L}_i = I_{asse} \overrightarrow{\omega}$\\
I card: $\sum \overrightarrow{F}_{est} = \frac{d\overrightarrow{p}_T}{dt} = m_{tot} \overrightarrow{a}_{CM}$\\
II card: $\sum \overrightarrow{\tau}_{est} = \frac{d\overrightarrow{L}_T}{dt}$

\vspace{-3mm}
\subsection*{Urti}
Elastico: conserva energia e qta moto\\
Anelastico: conserva qta moto\\
Compl. anelastico: corpi si uniscono\\
Compl. anelastico: $v_f = \frac{m_1 v_1 + m_2 v_2}{m_1 + m_2}$ solo q\\
Elastico:\\
$\left\{ \begin{array}{ll} m_1 v_{1i} + m_2 v_{2i} = m_1 v_{1f} + m_2 v_{2f} \\ m_1 v_{1i}^2 + m_2 v_{2i}^2 = m_1 v_{1f}^2 + m_2 v_{2f}^2 \end{array} \right.$

\vspace{-3mm}
\subsection*{Moto armonico}
$x(t) = A \cos(\omega t + \phi)$\\
$v(t) = -\omega A \sin(\omega t + \phi)$\\
$a(t) = -\omega^2 A \cos(\omega t + \phi) = -\omega^2 x(t)$\\
$A = \sqrt{x_0^2 + \left(\frac{v_0}{\omega}\right)^2}$\\
$\phi = \tan^{-1} \left(-\frac{v_0}{\omega x_0}\right)$\\
$f = \frac{\omega}{2\pi}$ \quad $T = \frac{1}{f} = \frac{2\pi}{\omega}$\\
Molla: $\omega = \sqrt{\frac{k}{m}}\leftarrow$ pulsazione\\
Molle in parallelo: $k_{eq} = \sum_i k_i$\\
Molle in serie: $\frac{1}{k_{eq}} = \sum_i \frac{1}{k_i}$\\
Pendolo: $\omega = \sqrt{\frac{g}{l}}$

\vspace{-3mm}
\subsection*{Momenti di inerzia notevoli}
Anello intorno ad asse: $I = MR^2$\\
Cilindro pieno intorno ad asse: $I = \frac{1}{2}MR^2$\\
Cilindro cavo intorno ad asse: $I = MR^2$\\
Sbarra sottile, asse CM: $I = \frac{1}{12}ML^2$\\
Sbarra sottile, asse perpendicolare estremo: $I = \frac{1}{3}ML^2$\\
Sfera piena, asse CM: $I = \frac{2}{5}MR^2$\\
Sfera cava, asse CM: $I = \frac{2}{3}MR^2$\\
Lastra rettangolare, asse CM: $I = \frac{1}{12}M(a^2 + b^2)$

\vspace{-3mm}
\subsection*{Gravitazione}
3° legge di Keplero: $T^2 = \frac{4\pi^2}{GM}R^3$\\
Vel. fuga: $v = \sqrt{\frac{2GM}{R}}$\\
Mom. angol.: $L = mvr$\\
F. Gravitaz.: $F = \frac{GMm}{r^2}$\\
Pot. gravitaz.: $U = -\frac{GMm}{r}$\\
En. Mecc.: = $E_m = E_k + U = \frac{1}{2}mv^2 - \frac{GMm}{r}$\\
En. Meccanica si conserva.\\
Un punto materiale soggetto a forze centrali conserva anche momento angolare.\\
Velocità per arrivare a orbita geostaz:\\
$v = \sqrt{\frac{GM_T}{R_T}}$\\
Velocità fuga:\\
$v = \sqrt{\frac{2GM_T}{R_T}}$\\
Orbita geostaz:\\
$R_G = \sqrt[3]{\frac{GMT^2}{4\pi^2}}$ terra: $\sqrt[3]{\frac{gR_T^2T^2}{4\pi^2}}$ ca 36k km\\
$T = 24h = 86400s$

\vspace{-3mm}
\subsection*{Fluidi}
Spinta archimende: $B_A = \rho_L V g$\\
Continuità: $A v = cost$\\
Bernoulli: $P + \frac{1}{2}\rho v^2 + \rho g h = cost$\\
$p(h) = p_0 + \rho g h$
$\mathcal{V} = \pi r^2 v$

\vspace{-3mm}
\subsection*{Esercizi}
Velocità minima giro completo FUNE:\\
$v_{min} = \sqrt{5gL}$\\
Velocità minima giro completo ASTA:\\
$v_{min} = \sqrt{4gL}$\\
Forza interazione tra due corpi:\\
$F = \frac{m_2}{m_1 +m_2}F$\\
Equilibrio corpo rigido:\\
$\left\{ \begin{array}{ll} R_x = 0 \\ R_y = 0 \\ \mathcal{M}_{A_z} = 0 \end{array} \right.$

\vspace{-3mm}
\section*{Termodinamica}
\vspace{-3mm}
\subsection*{Primo principio}
Calore e cap. termica: $Q = mc \Delta T$\\ %verifica todo
Calore latente di trasf.: $L_t = \frac{Q}{m}$\\
Lavoro \underline{sul} sistema: $dW = -PdV$\\
En. interna: $dU = dQ - dW$\\
Entropia: $\Delta S_{AB} = \int_A^B \frac{dQ_{REV}}{T}$

\vspace{-3mm}
\subsection*{Calore specifico}
Per unità di massa $c = \frac{C}{m}$\\
gas perfetto: $c_p - c_v = R$\\
\textbf{Gas} \quad \textbf{\(c_V\)} \quad \textbf{\(c_p\)} \quad \ \ \textbf{\(\gamma\)} \\
\textbf{Mo} \quad \(\frac{3}{2}R\) \quad \(\frac{5}{2}R\) \quad \(\frac{5}{3}\) \\
\textbf{BI}  \quad \ \(\frac{5}{2}R\) \quad \(\frac{7}{2}R\) \quad \(\frac{7}{5}\) \\


\vspace{-3mm}
\subsection*{Gas perfetti}
Eq. stato: $pV = nRT = NkT$\\
Energia interna: $\Delta U = n c_V \Delta T$\\
Entropia: $\Delta S = n c_V \ln \frac{T_f}{T_i} + nR \ln \frac{V_f}{V_i}$\\
\underline{Isocora}:\\ $Q = \Delta U = n c_V \Delta T$ \quad $W = 0$\\
$\Delta S = n c_V \ln \frac{T_f}{T_i} = n c_V \ln \frac{p_f}{p_i}$\\
\underline{Isobara}:\\ $Q = n c_p \Delta T$ \quad $W = -p \Delta V$\\
$\Delta S = n c_p \ln \frac{T_f}{T_i} = nc_p \ln \frac{V_f}{V_i}$\\
\underline{Isoterma}:\\ $W = -Q = -nRT \ln \frac{V_f}{V_i}$\\
$\Delta S = nR \ln \frac{V_f}{V_i} = nR \ln \frac{p_i}{p_f}$\\
\underline{Adiabatica}:\\ $Q = 0$ \quad $p^{1-\gamma} T^\gamma = cost$\\
$W = -\Delta U = \frac{1}{\gamma - 1}(p_f V_f - p_i V_i)$ \quad $\Delta S = 0$\\
$pV^\gamma = cost$ \quad $TV^{\gamma - 1} = cost$

\vspace{-3mm}
\subsection*{Macchine termiche}
Efficienza: $\eta = \frac{W}{Q_H} = 1 - \frac{Q_C^-}{Q_H^+}$\\
Eff. Carnot: $\eta = 1 - \frac{T_C}{T_H}$\\
Teorema di Carnot: $\eta \leq \eta_{REV}$\\
C.O.P. frigorifero: $\epsilon = \frac{Q_C}{W}$\\
C.O.P. pompa di calore: $\epsilon = \frac{Q_H}{W}$

\vspace{-3mm}
\subsection*{Espansione termica solidi}
$\beta = 3\alpha$\\
Lineare: $\Delta L= \alpha L_0 \Delta T$\\
Volumetrica: $\Delta V = \beta V_0 \Delta T$

\vspace{-3mm}
\subsection*{Entropia}
$\Delta S_u = \Delta S_s + \Delta S_g$\\
$\Delta S = \frac{m \lambda}{T_0} + mc_G \ln \frac{T_f}{T_i}\leftarrow$ solido + liq\\
$\Delta S = \frac{Q}{T} \leftarrow$ gas a T cost\\
$\Delta S = \sum \frac{Q_i}{T_i} \leftarrow$ gas\\
$\Delta S_U > 0$ irreversibile\\
$\Delta S_U = 0$ reversibile\\

\end{multicols*}
\chapter*{Teoria}
\begin{multicols*}{2}
\begin{enumerate} 
\item	ENUNCIARE I PRINCIPI DELLA TERMODINAMICA\\
Un corpo, sul quale non agiscano forze o la cui risultante sia nulla, permane: \\
In quiete sé già in quiete \\
In uno stato di moto rettilineo uniforme se in movimento\\
L’accelerazione di un corpo è direttamente proporzionale alla risultante delle forze a cui è sottoposto e inversamente proporzionale alla massa \\
Se un corpo A applica una forza su un corpo B allora anche B applica una forza su A pari direzione e modulo ma verso opposto

\item	DEFINIZIONE DI LAVORO\\
Si consideri un punto materiale soggetto a una forza F che compie uno spostamento infinitesimo ds; si può definire il lavoro infinitesimo come: {}\\
Il lavoro compiuto dalla forza F, lungo una curva $\gamma$, tra i punti AB si può esprimere come: $L_{AB}^\gamma = \int_{\gamma A}^{B} \vec{F} \cdot d\vec{s}$


\item Teorema Energia cinetica\\
Il lavoro compiuto dalla risultante delle forze che agisce su un punto è pari alla sua variazione di energia cinetica.\\	
$dL = \vec{F} \cdot d\vec{s} = (F_T \hat{\mu}_T+ F_N \hat{\mu}_N) \cdot (ds\hat{\mu}_T) = F_T ds = mads = m \frac{dv}{dt}ds = mvdv$\\
$L^\gamma_{AB} = \int_{A\gamma}^{B} mvdv = \frac{1}{2}mv^2_B - \frac{1}{2}mv^2_A = \Delta E_c$

\item Forza conservativa\\
Una forza si dice conservativa se:\\
$\int_{\gamma_1A}^{B} \vec{F} \cdot d\vec{s} = \int_{\gamma_2A}^{B} \vec{F} \cdot d\vec{s} \quad \forall \gamma_1, \gamma_2 | \gamma_1 \neq  \gamma_2$

\item DEFINIZIONE DI ENERGIA POTENZIALE\\
Si definisce Energia Potenziale quella funzione di Stato tale che: $L_{AB}^\gamma = -\Delta E_P$\\
È possibile introdurre l’energia potenziale solo per le forze conservative

\item DEFINIZIONE DI FORZA CENTRALE A SIMMETRIA SFERICA\\
Si definisce forza centrale a simmetria sferica una forza che, definita un’origine O e data la posizione R del punto materiale, ha modulo definito in funzione della distanza RO e direzione pari alla retta passante per O e R. {}

\item Forza gravitazionale\\
La forza gravitazionale è una forza centrale a simmetria sfeerica definita come:\\
$\vec{F_G} = -\frac{GMm}{r^2}\hat{\mu_r}$\\
$L_{AB}^\gamma = \int_{\gamma A}^{B} \vec{F_G} \cdot d\vec{r} = -\gamma m_1 m_2 \int_{\gamma A}^{B} \frac{1}{r^2} dr = \gamma m_1 m_2 \left(\frac{1}{r_B} - \frac{1}{r_A}\right)$\\
Per definizione di energia potenziale:\\
$L_{AB}^\gamma = - \gamma m_1 m_2 \left(- \frac{1}{r_B} + \frac{1}{r_A}\right)$\\
$E_P = -\frac{GMm}{r} + cost$ (per convenzione c = 0 in quanto $E_P^G(\infty) = 0$)

\item Energia potenziale forza Coloumb\\
La forza di Coloumb è una forza centrale a simmetria sferica definita come:\\
$\vec{F_C} = \frac{1}{4\pi \varepsilon_0} \frac{q_1 q_2}{r^2}\hat{\mu_r}$\\
$L_{AB}^\gamma = \int_{\gamma A}^{B} \vec{F_C} \cdot d\vec{r} = \frac{q_1q_2}{h\pi\varepsilon_0} \int_{\gamma A}^{B} \frac{1}{r^2} dr = \frac{q_1q_2}{4\pi\varepsilon_0} \left(-\frac{1}{r_B} + \frac{1}{r_A}\right)$\\
Per definizione di energia potenziale:\\
$E_P = \frac{1}{4\pi\epsilon_0} \frac{q_1q_2}{r} + cost$

\item Teorema conservazione energia meccanica\\
$L = \Delta E_c = E_c^F - E_c^I$\\ Per t. forze vive\\
$L = -\Delta U = E_P^I - E_P^F$\\ Per def energia potenziale\\
$E_c^F -E_c^I = E_P^I - E_P^F$\\
$E_c^F + E_P^F = E_c^I + E_P^I$\\
$E_M^F = E_M^I$

\item Conservazione momento angolare per un punto materiale soggetto a forza centrale\\
Calcolo il momento della forza:\\
$\vec{M}_o = \vec{F}_c \times \vec{r} = 0$\\
Il momento è anche definito come:\\
$\vec{M}_o = \frac{d\vec{L}}{dt}$\\
Ciò implica che il momento angolare è costante.

\item DIMOSTRARE CHE IL MOTO DI UNA PARTICELLA IN UN CAMPO DI FORZE CENTRALI È PIANO\\
Trattandosi di forze centrali, il momento angolare si deve conservare.\\
L=r x mv = k; per poter mantenere costante direzione e verso del prodotto vettoriale, il moto della particella non può cambiare piano (moto piano) né il senso di rotazione

\item In un campo di forze centrali la velocità areolare si conserva\\
$dA = \frac{1}{2}r (rd\theta) = \frac{1}{2}r^2 d\theta$\\ (area infinitesima)\\
$\frac{dA}{dt} = \frac{1}{2}r^2 \frac{d\theta}{dt} = \frac{1}{2}r^2 \omega$ (velocità areolare)\\
$\vec{L} = \vec{r} \times \vec{p} = mr^2 \omega$\\
$\frac{dA}{dt} = \frac{L}{2m}$ Visto che $L$ è costante, la velocità areolare è costante.

\item ENUNCIARE LE LEGGI DI KEPLERO\\
	a) La terra compie un orbita ellittica di cui il sole è uno dei due fuochi\\
	b) Il raggio vettore che unisce, virtualmente, sole e terra, spazia aree uguali in tempi uguali (velocità areolare costante)\\
	c) I quadrati dei tempi che i pianeti impiegano a percorrere le loro orbite sono proporzionali al cubo del semiasse maggiore $\to$ $T^2 = \frac{4\pi^2}{GM}a^3$

\item SIGNIFICATO FISICO DEL SEGNO DELL’ENERGIA MECCANICA DI UN CORPO SOGGETTO SOLAMENTE ALLA FORZA GRAVITAZIONALE \\
Essendo soggetto a forze conservative si deve conservare l’energia meccanica: {}\\
L’energia cinetica non può che essere sempre positiva; l’unico termine che può divenire negativo è l’energia potenziale gravitazionale.\\
Se l’energia meccanica del corpo è positiva non è necessario che il corpo torni indietro, compiendo un’orbita chiusa, per mantenere il segno, cosa che accade in caso di energia meccanica negativa


\item 1° e 2° eq. cardinale dinamica sistemi\\
\begin{itemize}
\item $\vec{F}^E = \frac{d\vec{p}}{dt}$\\
Dimostrazione:\\
In un sistema di N particelle, ogni particella risente di una forza:
$\vec{F}_i = \vec{F}_i^E + \vec{F}_i^I$\\
Considerando il sistema nel suo complesso, deve valere il 2° principio della dinamica:\\
$\sum_{i=1}^N \vec{F}_i = \sum_{i=1}^N m_i \vec{a}_i$\\
$\sum_{i=1}^N \vec{F}_i^E +\vec{F}_i^I = \sum_{i=1}^N m_i \vec{a}_i$\\
Per il 3° principio della dinamica posso affermare:\\
$\sum_{i=1}^N \vec{F}_i^I = 0$\\
Quindi:
$\sum_{i=1}^N \vec{F}_i^E = \sum_{i=1}^N m_i \vec{a}_i$\\
$\vec{F}^E = \sum_{i=1}^N m_i \frac{d\vec{v}_i}{dt} = \frac{d\vec{p}}{dt}$

\item $\frac{d\vec{L}_0}{dt} = -\vec{v}_i \times m_i \vec{v}_{CM} + \vec{M}^E$\\
$\vec{L}_0 = \sum_{i=1}^N m_i \vec{r}_i \times \vec{v}_i$\\
$\frac{d\vec{L}_0}{dt} = \sum_{i=1}^N (\frac{d\vec{r}_i'}{dt} \times m_i \vec{v}_i + \vec{r}_i \times \frac{d\vec{v}_1}{dt})$\\
Tuttavia è noto che: $\vec{r}_i' = \vec{r}_i - \vec{r}_0$\\
Quindi: $\frac{d\vec{L}_0}{dt} = - \vec{v}_0 \times M \vec{v}_{CM} +\vec{M}^E$
\end{itemize}

\item Teoremi centro di massa\\
Definizione del luogo geometrico noto come centro di massa:\\
$\vec{r}_{CM} = \frac{1}{M} \sum_{i=1}^N m_i \vec{r}_i$\\
1° Teorema del centro di massa:\\
$\vec{v}_{CM} = \frac{d\vec{r}_{CM}}{dt} = \frac{1}{M} \sum_{i=1}^N m_i \vec{v}_i = \frac{\vec{p}}{M}$\\
2° Teorema del centro di massa:\\
$\vec{a}_{CM} = \frac{\vec{F}^E}{M}$ \quad dove $\vec{F}^E$ è la risultante delle forze esterne\\


\item T. di König\\
$E_c = E_c^{CM} + E_c^I$ con $E_c^{CM}$ energia cinetica del centro di massa e $E_c^I$ energia cinetica del sistema rispetto al centro di massa\\
Considero la velocità i-esima di una particella: $\vec{v}_i = \vec{v}_{CM} + \vec{v}_i'$, essa è esprimibile in funzione del sistema di riferimento del centro di massa:\\
$E_c = \sum_{i=1}^{N} \frac{1}{2} v_i^2 = \sum_{i=1}^{N} \frac{1}{2} (v_{CM} + v_i')^2 m = \sum_{i=1}^{N} \frac{1}{2} v_{CM}^2 m+ \sum_{i=1}^{N} \frac{1}{2} v_i'^2 m+ \sum_{i=1}^{N} v_i' v_{CM} m$

\item T. dell'Impulso\\
Definito impulso di una forza $\vec{I} = \int_{t_1}^{t_2} \vec{F} dt$\\
Il teorema dell'impulso afferma che:\\
$\vec{I} = \Delta \vec{p}$\\
Dimostrazione:\\
$\vec{I} = \int_{t_1}^{t_2} \vec{F} dt = \int_{t_1}^{t_2} \frac{d\vec{p}}{dt} dt = \int_{t_1}^{t_2} d\vec{p} = \Delta \vec{p}$

\item Urti\\
a) urto elastico: urto nel quale si mantiene sia l’energia cinetica sia la quantità di moto\\
B) urto anaelastico: urto nel quale si conserva solo la quantità di moto \\
c) urto completamente anaelastico: caso particolare di urto anaelastico dove i corpi rimangono “attaccati” dopo l’urto e si ha la massima perdita di energia cinetica

\item DIMOSTRAZIONE IN UN URTO COMPLETAMENTE ANAELASTICO SI HA LA MASSIMA PERDITA DI ENERGIA CINETICA\\
Per il teorema di Koning sull’energia cinetica:\\$E_c = E_c' + \frac{1}{2} M v_{CM}^2$\\
Negli urti anaelastici si conserva la quantità di moto, per il 1° teorema del centro di massa:\\
$\vec{p} = + M \vec{v}_{CM} = cost \rightarrow \vec{v_{CM}} = cost$\\
Ne consegue che l’unico modo che l’energia cinetica ha le variare è che $E_c' \to 0$ ; si ha dunque la massima perdita di energia cinetica

\item ENUNCIARE IL PRIMO PRINCIPIO DELLA TERMODINAMICA\\
La variazione di energia interna di un sistema è pari alla differenza tra calore scambiato e lavoro compiuto $\Delta U = Q - L$

\item DEFINISCI CALORE MOLARE\\
Si definisce calore molare il rapporto tra la capacità termica e il numero di moli di un sistema\\
$C_m = \frac{1}{n} \frac{dQ}{dT}$

\item DEFINISCI IL RENDIMENTO DI UNA MACCHINA TERMICA \\
Si definisce rendimento di una macchina termica il rapporto tra il lavoro compiuto dalla macchina ed il calore da essa assorbito $\eta = \frac{L}{Q_H}$

\item ENUNCIATI, E DIMOSTRAZIONE DELLA LORO EQUIVALENZA, DEL SECONDO PRINCIPIO DELLA TERMODINAMICA \\
a) ENUNCIATO DI CLAUSIUS\\
è impossibile realizzare una qualsiasi trasformazione che abbia come unico risultato il passaggio di calore da un corpo freddo ad un corpo caldo\\
b) ENUNCIATO DI KELVIN PLANK\\
è impossibile realizzare una qualsiasi trasformazione che abbia come risultato quello di convertire completamente in lavoro il calore prelevato da un solo termostato

\item ENUNCIARE E DIMOSTRARE IL TEOREMA DI CARNOT\\
a) tra le macchine termiche cicliche che operano tra due temperature ben definite hanno rendimento massimo quelle reversibili ossia quelle di carnot\\
b) tutte le macchine di carnot che operano tra due termostati T1 e T2 hanno lo stesso rendimento \\
c) il rendimento di una macchina di carnot dipende solo dalle temperature dei termostati tra i quali opera

\item ENUNCIATO E DIMOSTRAZIONE DEL TEOREMA DI CLAUSIUS\\
Considerando una macchina termica che opera agendo su un numero n di termostati e compie un lavoro Lm sull’ambiente

\vspace{10mm}
Considero ora che su questo n termostati operino n macchine di Carnot operanti tra un termostato T0 e un n-esimo termostato della macchina M.\\

\vspace{20mm}
Considero la macchina termica equivalente:\\

\vspace{10mm}
Essendo una macchina ciclica monoterma è necessario che\\ 
$Q_T = \sum Q_i \leq 0$\\
Per ogni macchina di Carnot è valida la relazione \\
$\frac{Q_i}{T_0} + \frac{Q_i}{T_i} = 0 \Rightarrow Q_i = t_0 \frac{Q_i}{Q_0}$\\
$Q_T = \sum Q_i = T_0 \sum \frac{Q_i}{T_i} \leq 0$\\
Siccome la macchina è reversibile devono valere nello stesso tempo le seguenti condizioni:\\
\begin{equation*}
\begin{cases}
Q_T = T_0 \sum \frac{Q_i}{T_i} \leq 0\\
Q_T = T_0 \sum \frac{Q_i}{T_i} \geq 0
\end{cases} \Rightarrow Q_T = 0
\end{equation*}

Nella situazione più generale in cui la trasformazione ciclica sia accompagnata da scambi di calore con una successione, di corpi a diversa temperatura T, si può assumere che il sistema interagisca con una distribuzione continua di serbatoi con ciascuno dei quali può scambiare o meno una quantità infinitesima di calore dQ; mediante integrazione \\
$\oint \frac{dQ}{T} \leq 0$ $\rightarrow$ integrale di Clausius

\item ENUNCIATO E DIMOSTRAZIONE DEL PRINCIPIO DELL’AUMENTO DELL’ENTROPIA\\
Definita la funzione di Stato entropia come:\\
$DeltaS = \int \frac{dQ}{T}$\\
Considerando un sistema termicamente isolato è sempre vero che:\\
$dQ = 0$\\
Dunque non può essere vero altro che:\\
$\Delta S \geq 0$ con $\Delta S = 0$ per trasformazioni reversibili\\
Il sistema isolato per eccellenza è l’universo termodinamico; nella realtà trasformazioni perfettamente reversibili non esistono. Dunque l’entropia dell’universo tende sempre ad aumentare.

\item UTILIZZO DEL PRINCIPIO DI AUMENTO DELL’ENTROPIA NEL SECONDO PRINCIPIO DELLA TERMODINAMICA \\
Considerando una macchina ciclica che scambia calore con un solo termostato; la variazione di entropia dell’universo di tale ciclo è pari a quella del termostato:\\
$\Delta S_U = -\frac{Q}{T}$
Tuttavia l’entropia deve essere $\Delta S \geq 0$ ciò vuol dire quindi che:\\
$Q\leq0$\\
Confermando nuovamente la validità del l’enunciato di Kelvin 

\item UTILIZZO DEL PRINCIPIO DI AUMENTO DELL’ENTROPIA NELLA DIMOSTRAZIONE DEL TEOREMA DI CARNOT\\
Considero una macchina termica qualsiasi che assorba calore $Q>0$ e che compia lavoro sull’ambiente $L > 0$ cedendo il calore $L-Q$ a un serbatoio freddo; la variazione di entropia dell’universo coincide con quella delle sorgenti:\\
$\Delta S_U = \Delta S_1 + \Delta S_2 = \frac{-Q}{T_1} + \frac{-Q}{T_2} = Q \left(\frac{1}{T_2} - \frac{1}{T_1}\right) - \frac{L}{T_2}$\\
$L = Q (1 - \frac{T_2}{T_1}) - T_2 \Delta S_U$\\
$\eta = \frac{L}{Q} = \left(1 - \frac{T_2}{T_1}\right) - \frac{T_2}{Q} \Delta S_U$\\
È quindi stato identificato il rendimento di una macchina termica e confermato che la macchina di Carnot è la più efficace possibile.\\

\item Equazione differenziale moto armonico\\
$\frac{d^2x}{dt^2} x(t) + \omega^2 x(t) = 0$

\end{enumerate}
\end{multicols*}
\end{document}
