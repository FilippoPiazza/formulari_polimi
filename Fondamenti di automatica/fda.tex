\documentclass[10pt,landscape, a4paper]{scrartcl} %article
\usepackage{calc}
\usepackage{ifthen}
\usepackage[landscape]{geometry}
\usepackage{amsmath,amsthm,amsfonts,amssymb}
\usepackage{color,graphicx,overpic}
\usepackage{hyperref}
\usepackage{mdframed}
\usepackage{pgfplots}			% grafici
\usepackage{imakeidx}			% indice			
\usepackage{array}				% matrici
\usepackage{mathtools}			% matematica		
\usepackage{chngcntr}			% counter per gli esercizi
\usepackage{multicol}			% colonne multiple
\usepackage[italian]{babel}


% Impostazioni grafici pgfplots, altrimenti si lamenta
\pgfplotsset{compat=1.18}

\mathtoolsset{showonlyrefs}

% Definizione dell'ambiente "Definizione"
\newtheorem{defn}{Definizione}
\newenvironment{definition}{\begin{mdframed}[backgroundcolor=Ivory2]\begin{defn}}{\end{defn}\end{mdframed}}

% Definizione dell'ambiente "Teorema"
\newtheorem{teorema}{Teorema}
\newenvironment{thm}{\begin{mdframed}[backgroundcolor=white]\begin{teorema}}{\end{teorema}\end{mdframed}}

% Definizione dell'ambiente "Dimostrazione"
\newtheorem{demnstrn}{Dimostrazione}
\newenvironment{dimostrazione}{\begin{mdframed}[backgroundcolor=white]\begin{demnstrn}}{\end{demnstrn}\end{mdframed}}



\pdfinfo{
  /Title (example.pdf)
  /Creator (TeX)
  /Producer (pdfTeX 1.40.0)
  /Author (Seamus)
  /Subject (Example)
  /Keywords (pdflatex, latex,pdftex,tex)}

% This sets page margins to .5 inch if using letter paper, and to 1cm
% if using A4 paper. (This probably isn't strictly necessary.)
% If using another size paper, use default 1cm margins.
\ifthenelse{\lengthtest { \paperwidth = 11in}}
    { \geometry{top=.5in,left=.5in,right=.5in,bottom=.5in} }
    {\ifthenelse{ \lengthtest{ \paperwidth = 297mm}}
        {\geometry{top=0.5cm,left=.5cm,right=.5cm,bottom=.5cm} }
        {\geometry{top=.5cm,left=.5cm,right=.5cm,bottom=.5cm} }
    }

% Turn off header and footer
\pagestyle{empty}

% Redefine section commands to use less space
\makeatletter
\renewcommand{\section}{\@startsection{section}{1}{0mm}%
                                {-1ex plus -.5ex minus -.2ex}%
                                {0.5ex plus .2ex}%x
                                {\normalfont\large\bfseries}}
\renewcommand{\subsection}{\@startsection{subsection}{2}{0mm}%
                                {-1explus -.5ex minus -.2ex}%
                                {0.5ex plus .2ex}%
                                {\normalfont\normalsize\bfseries}}
\renewcommand{\subsubsection}{\@startsection{subsubsection}{3}{0mm}%
                                {-1ex plus -.5ex minus -.2ex}%
                                {1ex plus .2ex}%
                                {\normalfont\small\bfseries}}
\makeatother

% Define BibTeX command
\def\BibTeX{{\rm B\kern-.05em{\sc i\kern-.025em b}\kern-.08em
    T\kern-.1667em\lower.7ex\hbox{E}\kern-.125emX}}

% Don't print section numbers
\setcounter{secnumdepth}{0}


\setlength{\parindent}{0pt}
\setlength{\parskip}{0pt plus 0.5ex}

%My Environments
\newtheorem{example}[section]{Example}
% -----------------------------------------------------------------------

\begin{document}
\raggedright
\footnotesize
\begin{multicols*}{3}


% multicol parameters
% These lengths are set only within the two main columns
%\setlength{\columnseprule}{0.25pt}
%\setlength{\premulticols}{1pt}
\setlength{\postmulticols}{1pt}
\setlength{\multicolsep}{1pt}
\setlength{\columnsep}{2pt}

\section*{Studio sistema}
\subsection{Classificazione}
\begin{enumerate}
	\item Numero di input e output: SISO, MISO, SIMO, MIMO
	\item Strettamente proprio: $y(t) = g(x(t), t)$ovvero non compare $u$ null'uscita.
	\item Proprio: $y(t) = g(u(t), t)$
	\item Tempo invariante (stazionario): $y(t) = g(u(t))$ non dipende da $t$. Altrimenti tempo variante.
	\item Lineare: combinazioni lineari di $u$ e $x$. No esponeziali, prodotti o funzioni. altrimenti non lineare.
	\item Dinamico: dipende da $x$. Ha ordine $n$ numero di $\dot x_n$. Altrimenti statico.
	\item Ordine: numero / dimensione ingressi.
\end{enumerate}

\subsection*{Sistemi a tempo continuo}
Ovvero nella forma $\dot x = Ax + Bu$, $y = Cx + Du$.

\subsubsection*{Equilibrio}
Cerco $y_e$ punto di equilibrio dato $u_e$ ingresso.
\begin{enumerate}
	\item Pongo le derivate a zero. $\dot x = 0$
	\item Risolvo il sistema lineare. Trovo $x_1, x_2, \ldots, x_n$
	\item Sostituisco nelle equazioni di output.
\end{enumerate}
Trovo $y_e$ punto di equilibrio.

\subsubsection*{Linearizzazione}
Scrivere equazione del sistema linearizzato attorno a un punto di equilibrio dato: $(\overline{x}, \overline{u})$\\
Scrivo il sistema in forma matriciale. Avrò $A, B, C, D$ dove $A$ è $n \times n$, $B$ è $n \times m$, $C$ è $p \times n$ e $D$ è $p \times m$. In genere $p = m = 1$:
\begin{equation}
	A = \left. \frac{\partial f}{\partial x} \right|_{(\overline{x}, \overline{u})} \quad B = \left. \frac{\partial f}{\partial u} \right|_{(\overline{x}, \overline{u})} \quad C = \left. \frac{\partial g}{\partial x} \right|_{(\overline{x}, \overline{u})} \quad D = \left. \frac{\partial g}{\partial u} \right|_{(\overline{x}, \overline{u})}
\end{equation}
Avrò quindi:
\begin{equation}
	\begin{cases}
		\delta \overline{x} = A\delta \overline{x} + B\delta u\\
		\delta \overline{y} = C\delta \overline{x} + D\delta u
	\end{cases}
\end{equation}

\subsubsection*{Studio stabilità}
A partire dal sistema linearizzato.
Studio autovalori di $A$. Se hanno tutti parte reale negativa, il sistema è asintoticamente stabile.

\subsubsection*{Funzione di trasferimento}
Due modi per trovarla:
\begin{equation}
	G(s) = C(sI - A)^{-1}B + D
\end{equation}
Oppure:\\
Sostituisco le derivate $\dot x_i$ con $sX_i$ e risolvo il sistema lineare: $y = G(s)u$.\\
Il sistema si risolve in genere per sostituzione.\\
NB: (se c'è un $\alpha$) se serve valutare oss. e ragg. bisogna controllare gli autovalori positivi e cercare di eliminarli (semplificare numeratore e denominatore).

\subsubsection*{Studio ossevabilità e raggiungibilità}
Osservabilità:\\
\begin{equation}
	M_O = \left[[C] [A\cdot C] [A^{2}\cdot C]\cdots [A^{n-1} \cdot C]\right]^T
\end{equation}
(in verticale)\\
Se il determinante di $M_O$ è diverso da zero, il sistema è ossevabile.\\

Raggiungibilità:\\
\begin{equation}
	M_R = \left[[B] [A\cdot B] [A^{2}\cdot B] \cdots [A^{n-1} \cdot B]\right]
\end{equation}
Se il determinante di $M_R$ è diverso da zero, il sistema è raggiungibile.\\

Per entrambi, se il n. poli = ordine, allora il sistema è ossevabile e raggiungibile.

\subsubsection*{Guadagno statico}
Studia stabilità. Se as. stabile, il guadagno statico è il rapporto tra l'uscita e l'ingresso a regime, ovvero il $G(s)$ in $y = G(s)u$.

\subsubsection*{Calcolo risposta analitica (modi)}
Risposta libera significa $u = 0$. L'esercizio da uno stato $x(0) = (x_1, x_2, \ldots, x_n)$.\\
\begin{enumerate}
	\item Calcolo autovalori di $A$.
	\item Calcolo autovettori di $A$.
	\item Calcolo $x(t) = \sum_{i=1}^{n} x_i e^{\lambda_i t} v_i$ ovvero la somma dei modi.
	\item Calcolo $y(t) = Cx(t)$.
\end{enumerate}


% \subsubsection*{Routh}
% Partendo dall'equazione caratteristica: $a_ns^n + a_{n-1}s^{n-1} + \ldots + a_1s + a_0 = 0$\\
% Costruisco la tabella di Routh, alternando i coefficienti $a$ sulle prime due righe.

% \begin{equation}
% 	\begin{array}{c|c|c|c|c|c}
% 		n 		& a_n 		& a_{n-2} 	& a_{n-4} 	& a_{n-6} 	& \ldots\\
% 		n-1 	& a_{n-1} 	& a_{n-3} 	& a_{n-5} 	& a_{n-7} 	& \ldots\\ \hline
% 		n-2 	& x_{n-2,1} & x_{n-4,2} & x_{n-6,3} & x_{n-8,4} & \ldots\\
% 		n-3 	& x_{n-3,1} & x_{n-5,2} & x_{n-7,3} & x_{n-9,4} & \ldots\\
% 		\vdots	& \vdots 	& \vdots 	& \vdots 	& \vdots 	& \vdots\\
% 		2		& x_{2,1} 	& x_{0,2} 	& \ldots 	& 			& 		\\
% 		1		& x_{1,1} 	& \ldots 	& 			& 			& 		\\
% 		0		& x_{0,1} 	& 		 	& 			& 			& 
% 	\end{array}
% \end{equation}
% Per valutare ogni elemento alle righe successive, uso questa formula:
% \begin{equation}
% 	x_{i,h} = - \frac{1}{x_{i+1,1}} \cdot \begin{vmatrix} x_{i+2,1} & x_{i+2,h+1} \\ x_{i+1,1} & x_{i+1,h+1} \end{vmatrix}
% \end{equation}
% Se in un una riga trovo un elemento nullo, considero nulli tutti i successivi e passo alla riga seguente.\\
% Il risultato finale è una tabella triangolare di righe a lunghezza decrescente. L'ultima ha un solo elemento.\\
% Se tutti gli elementi sono diversi da zero, la tabella è ben definita. Se non è ben definita, il sistema è instabile.\\
% Il sistema è stabile se tutti gli elementi della prima colonna hanno lo stesso segno.

\subsection*{Sistemi a tempo discreto}
Ovvero nella forma $x(k+1) = Ax(k) + Bu(k)$, $y(k) = Cx(k) + Du(k)$.

\subsubsection*{Equilibrio}
Cerco $y_e$ punto di equilibrio dato $u_e$ ingresso.
\begin{enumerate}
	\item Pongo $x(k+1) = x(k)$
	\item Risolvo il sistema lineare. Trovo $x_1, x_2, \ldots, x_n$
	\item Sostituisco nelle equazioni di output.
\end{enumerate}
Trovo $y_e$ punto di equilibrio.

\subsubsection*{Studio stabilità}
A partire dal sistema linearizzato.
Studio autovalori di $A$. Se hanno tutti parte reale $< 1$, il sistema è asintoticamente stabile.

\subsubsection*{Linearizzazione}
Scrivere equazione del sistema linearizzato attorno a un punto di equilibrio dato: $(\overline{x}, \overline{u})$\\
Scrivo il sistema in forma matriciale. Avrò $A, B, C, D$ dove $A$ è $n \times n$, $B$ è $n \times m$, $C$ è $p \times n$ e $D$ è $p \times m$. In genere $p = m = 1$:
\begin{equation}
	A = \left. \frac{\partial f}{\partial x} \right|_{(\overline{x}, \overline{u})} \quad B = \left. \frac{\partial f}{\partial u} \right|_{(\overline{x}, \overline{u})} \quad C = \left. \frac{\partial g}{\partial x} \right|_{(\overline{x}, \overline{u})} \quad D = \left. \frac{\partial g}{\partial u} \right|_{(\overline{x}, \overline{u})}
\end{equation}
Avrò quindi:
\begin{equation}
	\begin{cases}
		\delta x(k+1) = A\delta x + B\delta u\\
		\delta y = C\delta x + D\delta u
	\end{cases}
\end{equation}

\subsubsection*{Calcolo movimento libero uscita (modi)}
Movimento libero significa $u = 0$. L'esercizio da uno stato $x(0) = (x_1, x_2, \ldots, x_n)$.\\
\begin{enumerate}
	\item Calcolo autovalori di $A$.
	\item Calcolo autovettori di $A$.
	\item Calcolo $x(t) = \sum_{i=1}^{n} x_i \lambda_i^{k} v_i$ ovvero la somma dei modi.
	\item Calcolo $y(t) = Cx(t)$.
\end{enumerate}

\section*{Associare grafico al sistema}
\begin{enumerate}
	\item $Re(\lambda) = 0$ oscillazioni costanti.
	\item $Re(\lambda) > 0$ diverge.
	\item $Re(\lambda) < 0$ converge quindi studio tempo dominante e assestamento (vedi andamento qualitativo).
\end{enumerate}


Il numeratore influenza l'ampiezza della risposta.
\begin{enumerate}
	\item poli reali negativi: risposta smorzata esponenzialmente.
	\item poli complessi coniugati: oscillazioni smorzate.
	\item Poli puramente immaginari: oscillazioni non smorzate.
	\item Poli reali positivi: risposta crescente esponenzialmente.
\end{enumerate}



\section*{Risposta analitica a impulso unitario (Heaviside)}
Dato un $G(s) = \frac{N(s)}{D(s)}$ con $N(s)$ e $D(s)$ polinomi.\\
\begin{enumerate}
	\item Sviluppo di $G(s)$ in fratti semplici: (NB i poli uguali appaiono una volta per ogni molteplicità)
	\begin{equation}
		G(s) = \frac{N(s)}{D(s)} = \sum_{i=1}^{n} \frac{P_i}{(s - \lambda_i)}
	\end{equation}
	\item Calcolo i $P_i$: Sostituisco $s = \lambda_i$ in $\frac{N(s)}{Q(s)}$ dove $Q(s)$ è il polinomio che rimane dopo aver rimosso il termine $(s - \lambda_i)$ da $D(s)$.
	\item Se non si può calcolare come sopra, eguaglio $\sum_{i=1}^{n} P_i\cdot Q_i(s) = N(s)$ e risolvo il sistema lineare e sostituisco $s$ tale che sia diverso da tutti i $\lambda_i$.
	\item Sostituisco i $P_i$ in $G(s)$.
	\item Calcolo la trasformata inversa di $G(s)$, ovvero i termini del tipo $\frac{P_i}{s - \lambda_i} \rightarrow P_i\cdot e^{\lambda_i t}$.
\end{enumerate}
\begin{equation}
	f(t) = \left(\sum_{i=1}^{n} P_i\cdot e^{\lambda_i t}\right)sca(t)
\end{equation}





\section*{Schemi a blocchi}
Serie: $G(s) = G_1(s)\cdot G_2(s)$\\
Parallelo: $G(s) = G_1(s) + G_2(s)$\\
Retroazione negativa: $G(s) = \frac{G_1(s)}{1 + G_1(s)G_2(s)}$\\
Retroazione positiva: $G(s) = \frac{G_1(s)}{1 - G_1(s)G_2(s)}$\\
Per lo studio di stabilità, si possono ignorare i blocchi in retroazione.

\section*{Andamento qualitativo}
Data una funzione di trasferimento $G(s)$\\
Il polo dominante è quello con la parte reale maggiore.\\
NB se negativo, il polo dominante ha il valore assoluto minore.\\
\begin{enumerate}
	\item valore finale: $y_{\infty} = \lim_{s \to 0} s\cdot G(s)$
	\item valore iniziale: $y(0) = \lim_{s \to \infty} s\cdot G(s)$
	\item derivata iniziale: $\dot y(0) = \lim_{s \to \infty} s^2\cdot G(s)$
	\item tempo dominante: $\tau = \left| \frac{1}{\Re(\lambda)}\right|$ dove $\lambda$ è il polo dominante.
	\item tempo di assestamento: al $99\%$ è $5\tau$ (sarebbe $4.6\tau$), al $95\%$ è $3\tau$.
	\item Sovraelongazione percentuale: $S\% = e^{\frac{-\xi \pi}{\sqrt{1 - \xi^2}}} = \frac{Max - y_\infty}{y_\infty} \cdot 100$\\
			Ovvero il punto di massimo in che percentuale sfora il valore finale.
	\item Se $\mu = e^{-k s}$ allora il ritardo $\tau_r = k$, quindi nel disegno la funzione parte da $k$.
	\item Studio gli zeri:
	 \begin{multicols*}{2}
		\begin{enumerate}
			\item zero positivo
			\item zero negativo più vicino all'origine del piano complesso rispetto ai poli
			\item zero negativo più lontano dall'origine del piano complesso rispetto ai poli
		\end{enumerate} \includegraphics*[width=1\columnwidth]{zeri.png}
	 \end{multicols*}

	\item Se ho poli complessi coniugati, studio $\xi = \frac{\Re(\lambda)}{|\lambda|}$
	    \begin{multicols*}{2}
			\begin{enumerate}
			\item $\xi < 0.5$
			\item $\xi = 0.5$
			\item $\xi > 0.5$
		\end{enumerate} \includegraphics*[width=1\columnwidth]{xi.png}
		\end{multicols*}
		
\end{enumerate}
Disegno il grafico con multipli di $\tau$ su asse $x$ i valori di $y$ su asse $y$.\\
Se chiede andamento qualitativo a $sca$ allora i valori finali e iniziali devono essere moltiplicati per $\frac{1}{s}$. Se $Ram$ allora per $\frac{1}{s^2}$.\\
Margine di fase: $\varphi_m = 180 + \arg(L(j\omega_\pi))$ (distanza da $-180^{\circ}$).\\
$\omega_c$ è il punto in cui il modulo di $L(j\omega)$ passa per $0 dB$.



\section*{Disegno diagrammi di Bode}
Viene fornita una funzione di trasferimento $G(s)$.
\begin{enumerate}
	\item trasformo la funzione nella forma: $G(s) = \frac{\mu}{s^m}\cdot \frac{\prod_{i=1}^{n} (s + z_i)}{\prod_{i=1}^{n} (s + p_i)}$
	\item trovo zeri, poli e $\mu$ guadagno statico.
	\item Per il diagramma di modulo:
		\begin{itemize}
			\item Popolo gli assi: in generale l'ordine di grandezza del primo polo o zero lo colloco nella terza decade. Su asse $y$ metto $\alpha_{dB}$ poco sopra la metà.
			\item Tipo di funzione: 0 se $m = 0$, 1 se $m = 1$, 2 se $m \geq 2$.
			\item Posizione iniziale: $\alpha_{dB} = 20\log_{10} (\mu)$.
			\item metto un punto in $(\lambda_1, \alpha_{dB})$ con $\lambda_1$ polo o zero più piccolo.
			\item Traccio una retta con pendenza $20\cdot m$ dB/decade verso sinistra.
			\item Proseguo verso destra e ogni polo o zero mi fa cambiare pendenza. Se zero sale, se polo scende.
		\end{itemize}
	\item Per il diagramma di fase:
		\begin{itemize}
			\item Copio asse $x$ da sopra, asse $y$ centro il valore di $\varphi$.
			\item fase iniziale $\varphi = - m \cdot 90 + k$ con $k = -180$ se $\mu < 0$, $k = 0$ altrimenti.
			\item Traccio una retta orizzontale fino al primo polo o zero.
			\item Ogni polo o zero mi fa cambiare di $90$ gradi. Se $z^+$ scende, se $p^+$ sale, se $z^-$ sale, se $p^-$ scende.
			\item Traccio curva reale approssimata. (per Nyquist)
					NB: se smorzamento $\xi < \frac{1}{sqrt{2}}$ allora sui cambi di modulo la funzione reale ha un picco di risonanza. (ovvero sfora)
		\end{itemize}
\end{enumerate}
P.S.: Se c'è complesso coniugato, i due poli sono la radice del coefficiente di $s^0$

\section*{Disegno diagrammi di Nyquist}
Diagramma polare. L'esercizio fornisce una funzione di trasferimento $G(s) = \frac{\mu}{s^g}\cdot \frac{\prod_{i=1}^{n} (s + z_i)}{\prod_{i=1}^{n} (s + p_i)}$.\\
Regole di tracciamento, dato diagramma di Bode:
\begin{itemize}
	\item Se $g = 0$ allora il punto di partenza è $(\mu,0)$.
	\item Se $g < 0$ allora il punto di partenza è l'origine.
	\item Se $g > 0$ allora il punto di partenza è l'infinito.
	\item La funzione abbandona l'asse reale sempre ortogonalmente.
	\item Se numero di poli - numero di zeri $>0$ allora il punto di arrivo è l'origine.
\end{itemize}
Per disegnare il diagramma, utilizzo la fase di Bode come angolo e il modulo come raggio (coordinate polari).

\section*{Criteri di stabilità} % bode, nyquist, piccola fase
Dato un sistema. Se bisogna scegliere, si fa Nyquist.

\subsection*{Criterio di Nyquist}
Il sistema è asintoticamente stabile se $N = P$ con $P$
numero di poli a parte reale positiva e $N$ numero di giri antiorari del diagramma di Nyquist.
Studio stabilità $\forall$ intervallo di $\frac{-1}{k}$ e poi passo a $k$.\\
Esempio:
$-0,5 < -\frac{1}{k} < 0 \Rightarrow k > 2$

\subsection*{Criterio di Bode}
Condizioni per poter applicare il criterio di Bode:
\begin{enumerate}
	\item Numero di poli a parte reale positiva è zero.
	\item $w_c$ è unica ovvero passa una sola volta per $0 dB$.
	\item La funzione di trasferimento è strettamente propria.
\end{enumerate}
Se è applicabile, il sistema è asintoticamente stabile se $\mu > 0$ e $\varphi_m > 0$.

\section*{Risposta in frequenza / transitorio esaurito}
Margine di guadagno: $MG = \frac{1}{|L(j\omega_\pi)|}$\\
Trasformo $u(t)$ in $y(t)$: i numeri diventano coeff. di $u$ e $\kappa sin(\alpha t + \beta)$ diventa $\kappa |G(j\alpha)|sin(\alpha t + \beta + \arg(G(j\alpha)))$. Analogamente per $cos$.\\
Posso calcolare $|G(j\alpha)|$ oppure prenderlo dal diagramma di Bode $10^{\frac{val}{20}}$\\
Attenzione a sovraelongazione.


\section*{Regolatore}
L'esercizio fornisce una funzione di trasferimento $G(s)$ con uno schema di controllo e delle specifiche.\\
Ogni specifica mi da un vincolo su $G(s)$.

\subsection*{Studio vincoli}
\begin{itemize}
	\item Attenuazione ampiezza $n(t) = sin(\omega_n\cdot t)$ inferiore a $-15 dB$.\\
		$|L(j\omega)| < -15 dB$.
	\item Attenuazione ampiezza $d(t) = sin(\omega_d\cdot t)$ inferiore a $-15 dB$.\\
		$|L(j\omega)| > 15 dB$.
	\item Banda passante in anello chiuso $B > 10 rad/s$.\\
		$\omega_c > 10$.
	\item Ampiezza segnale $y(t)$ a fronte di ingressi $d(t) = 5sin(\omega_d \cdot t)$ inferiore a $0.1$.\\
		$|S(j\omega)| \cdot 5 < 0.1$ ovvero $|L(j\omega)| > 50 dB$.
	\item Margine di fase superiore a $60^{\circ}$.\\
		$\varphi_m > 60^{\circ}$.
	\item Attenuazione $w(t) = sin (\omega_w \cdot t)$ superiore a $-3 dB$ $[0,20]$.\\
		$\omega_b \geq 20 \Rightarrow \omega_c > 20$.
	\item $|e(t)| < 0.11$ a trans.es.a fronte di $w(t) = (6 + 6sin(\omega_w \cdot t)) sca(t)$ e $n(t) = \frac{2}{6} cos(\omega_n \cdot t) sca(t)$ e $\omega_n \geq 1$
		\begin{enumerate}
			\item $6|S(j\omega)| < 0.1 \Rightarrow \frac{6}{|L(j\omega)|} < 0.1 \Rightarrow |L(j\omega)| > 60 dB$
			\item $\frac{2}{6}|F(j\omega)| < 0.01 \Rightarrow \frac{2}{6}|L(j\omega)| < 0.01 \Rightarrow |L(j\omega)| > \frac{3}{100} dB$
		\end{enumerate}		
	\item Sovraelongazione percentuale nella risposta a scalino inferiore al $25\%$\\
		$S\% < 25\% \Rightarrow e^{\frac{-\xi \pi}{\sqrt{1 - \xi^2}}} < 0.25$.\\
		$\xi \pi > ln(0.25) \cdot \sqrt{1 - \xi^2} \Rightarrow \xi > \frac{ln(0.25)}{\sqrt{\pi^2 + |ln(0.25)|^2}} \simeq 0.404$\\
		$\varphi_m > 40^{\circ}$.
	\item Se ho un vincolo del tipo $|e_\infty| = Ram(t)$ o $|e_\infty| = sca(t)$ allora l'ordine di $L(s)$ è $g+1$. 
	\item Tempo di assestamento inferiore a 2 unità di tempo.\\
		Se $\varphi_m > 75^{\circ}$ allora $\frac{1}{\omega_c} < 2$ quindi $\omega_c > 0.5$.\\
		Se $\varphi_m \leq 75^{\circ}$ allora $\frac{1}{\omega_c \cdot \xi} < 2$ quindi $\omega_c > \frac{0.5}{\xi}$ dato che $\xi = \frac{\varphi_m}{100}$ allora $\omega_c = 0.8$.
\end{itemize}
\begin{equation}
	\begin{array}{|c|c|c|c|}
	\hline
	|e_\infty|  & \frac{A}{S}    & \frac{A}{S^2}  & \frac{A}{S^3} \\ \hline
	g=0  & \frac{A}{1+u}  & \infty         & \infty       \\ \hline
	g=1  & 0              & \frac{A}{u}    & \infty       \\ \hline
	g=2  & 0              & 0              & \frac{A}{u}  \\ \hline
	\end{array}
\end{equation}
\begin{itemize}
	\item \textbf{NB} se ho due vincoli su $e_\infty$ ma uno è $sca$ e l'altro è $ram$, allora ignoro il $sca$.
	\item $|e_\infty| \leq 0.5$ con $w(t) = 8 Par(t)$\\
			$\frac{8}{u} < 0.5$ quindi $u > 16$ e $g=2$.
	\item $|e_\infty| \leq 0.5$ con $w(t) = 8 Sca(t)$\\
			$\frac{8}{1+u} < 0.5$ quindi $u > 15$ e $g=0$.
	\item $|e_\infty| \leq 0.5$ con $w(t) = 8 Ram(t)$\\
		$\frac{8}{u} < 0.5$ quindi $u > 16$ e $g=1$.
	\item $|e_\infty| = 0$ allora $g + 1$.
\end{itemize}

\subsection*{Progetto per inversione}
\begin{enumerate}
	\item Sul diagramma di Bode, disegno i vincoli ottenui dalle specifiche.
	\item Scelgo una $L(s)$ che rispetti i vincoli (disegnandola sul diagramma).
	\item Calcolo $R(s) = L(s) \cdot G(s) ^{-1}$.
\end{enumerate}

\end{multicols*}
\end{document}